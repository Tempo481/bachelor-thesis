%*******************************************************
% Abstract in German
%*******************************************************
\begin{otherlanguage}{ngerman}
	\pdfbookmark[0]{Zusammenfassung}{Zusammenfassung}
	\chapter*{Zusammenfassung}
	Mit Open Policy Agent (OPA) ist es möglich die Zugriffskontrolle (access control) von einem Server zu entkoppeln. Wenn ein Client eine Anfrage an einen Server sendet, evaluiert also dieser Server nicht selbst eine Zugriffsentscheidung, sondern OPA als externes Programm. Hier entsteht zusätzliche Kommunikation zwischen Server und OPA. In OAuth2 Systemen geschieht die Evaluierung von Zugriffsentscheidungen anhand von Token, die der Client an den Server sendet. In bisherigen Performancetests wurde die Latenz bei der Kommunikation zwischen Server und OPA nicht in Betrachtung gezogen, sowie keine Zugriffsentscheidungen (access decision) anhand von Token getroffen \citep{opaperformance:2021:07}.\smallskip

	Um zu untersuchen, inwiefern sich eine Entkopplung der Zugriffskontrolle mit OPA auf die Performance auswirkt, wurden zwei Testserver implementiert. In dem einen Server wird die Zugriffskontrolle mit OPA entkoppelt (decoupled), in dem anderen evaluiert der Server Zugriffsentscheidungen selbst. Um die Performance zu untersuchen, wurde Apache JMeter verwendet. Dieses Tool generiert Anfragen mit validem Token an die Server und misst die Response Time der Antworten. Bei zehn gleichzeitigen Nutzern stellte es sich heraus, dass die Response Times bei dem Server mit OPA durchschnittlich dreimal so hoch ausfielen im Vergleich zu dem Server ohne OPA. Zudem skaliert der Server mit OPA schlechter, denn bei ansteigender Last auf bis zu 100 Nutzer stiegen die Respone Times linear, während bei dem Server ohne OPA diese konstant niedrig blieben. Es konnte also ein signifikanter Performancenachteil nachgewiesen werden. 
	

\end{otherlanguage}

%*******************************************************
% Abstract in English
%*******************************************************
\pdfbookmark[0]{Abstract}{Abstract}


\begin{otherlanguage}{american}
	\chapter*{Abstract}

	Open Policy Agent (OPA) decouples access control from a server. When a client sends a request to a server, this server does not evaluate an access decision itself, but OPA does as an external program. This creates additional communication between the server and OPA. In OAuth2 systems, the evaluation of access decisions is based on tokens that the client sends to the server. In past performance tests, the latency of the communication between server and OPA was not taken into account, and no access decisions were made on the basis of tokens \citep{opaperformance:2021:07}.\smallskip

	In order to investigate to what extent decoupling the access control with OPA affects the performance, two test servers were implemented. In one server the access control is decoupled with OPA, in the other the server evaluates access decisions itself. Apache JMeter was used to examine the performance. This tool generates requests with valid tokens to the servers and measures the response times. With ten simultaneous users, response times on the server with OPA turned out to be three times as high compared to the server without OPA. In addition, the server with OPA scales worse, because with increasing load up to 100 users, the response times increased linearly, while with the server without OPA these remained consistently low. A significant performance disadvantage could be demonstrated.


\end{otherlanguage}

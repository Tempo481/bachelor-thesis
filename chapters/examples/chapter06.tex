\chapter{Zusammenfassung}
\label{Zusammenfassung}
Es wurde anhand von Performancetests mit Apache JMeter gezeigt, dass eine externe Zugriffskontrolle mit Open Policy Agent eine negative Auswirkung auf die Response Times und Verfügbarkeit des Servers sowie auf die CPU-Auslastung des Systems hat. Hierbei wurden mittels Apache JMeter Anfragen mit validem JSON Web Token auf die Server generiert. Diese Server sind nach der OAuth2 Spezifikation Ressource Server, da sie Zugriff auf die Schnittstellen nur zulassen, wenn ein valider Token vorliegt. Dieser JSON Web Token wird durch eine Authentifizierung eines Nutzers an einen Authorization Server erhalten. Die Ressource Server prüfen einerseits, ob der Token valide ist und andererseits ob in dem Token die notwendige Berechtigung enthalten ist, um einen Zugriff auf die Schnittstelle des Servers zuzulassen. Hier wurde eine rollenbasierte Zugriffskontrolle implementiert, das heißt der authentifizierte Nutzer muss einer bestimmten Rolle angehörig sein, um Zugriff auf die Schnittstelle zu erhalten. Diese Zugriffskontrolle wurde in beide Server auf unterschiedliche Weise realisiert. Denn um den Einfluss von externer Zugriffskontrolle mit Open Policy Agent auf die Performance zu untersuchen, musste ein Zweitserver als Vergleichsserver implementiert werden. Dieser setzt die Zugriffskontrolle in der Serverapplikation selbst um. Die anfängliche Hypothese, dass durch die zusätzlich entstehende Kommunikation zwischen Server und Open Policy Agent ein Performancenachteil entsteht, konnte bestätigt werden.\bigskip

Da dieser Performancenachteil signifikant ist und unter starker Last das System mit entkoppelter Zugriffskontrolle nicht stabil arbeitet, da der Server von Open Policy Agent sperrt, kann als weiterer Ausblick untersucht werden, inwiefern sich die Performance von Open Policy Agent verbessern lässt. Hier sollte zum einen die zusätzliche Latenz, die bei der HTTP-Kommunikation zwischen Server und Open Policy Agent entsteht, verbessert werden und die CPU-Auslastung, verringert werden. Denn selbst bei nur zehn gleichzeitigen Threads arbeitete das System mit Open Policy Agent am Anschlag. Außerdem könnte Open Policy Agent mit weiteren Lösungen zur externen Zugriffskontrolle verglichen werden. Hier ist als weitere Lösung zum Beispiel die Keycloak Client Adapter zu nennen \citep{keycloakclientadapter:2021}. Hier werden Zugriffsrichtlinien nicht programmiert, sondern können in dem graphischen Admin-Menü von Keycloak festgelegt werden. Ein Nachteil dieser Lösung hingegen ist, dass sich hier zum einen an Keycloak als Authorization Server gebunden wird und zum anderen sind Keycloak Client Adapter nicht für alle Programmiersprachen verfügbar im Gegensatz zu Open Police Agent, das praktisch system und plattformunabhängig ist. Die Lösung von Keycloak könnte unter den Kriterien der Performance, Funktionalität und Kompatibilität mit Open Policy Agent verglichen werden. Außerdem könnte untersucht werden, inwiefern sich die Performance mittels Kubernetes verbessern lässt, wenn mehr als ein Node in einem Cluster verwendet wird. Hierdurch kann die Last auf mehreren physikalischen Maschinen verteilt werden, was potenziell die Performance speziell in einem System mit entkoppelter Zugriffskontrolle merklich verbessern könnte, da sich Open Policy Agent als äußerst CPU-intensive Komponente herausgestellt hat. 

